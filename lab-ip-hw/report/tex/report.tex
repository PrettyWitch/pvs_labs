\documentclass[a4paper,12pt]{article}

\input{header.tex}

\title{Отчёт по лабораторной работе \\ <<IP-маршрутизация>>}
\author{Дун Юйхань}

\begin{document}

\maketitle

\tableofcontents

% Текст отчёта должен быть читаемым!!! Написанное здесь является рыбой.

\section{Топология сети}

Топология сети и использыемые IP-адреса показаны на рис.~\ref{fig:network}.

\begin{figure}
\centering
\includegraphics[width=\textwidth]{includes/network_gv.pdf}
\caption{Топология сети}
\label{fig:network}
\end{figure}


\section{Назначение IP-адресов}

Ниже приведён файл настройки протокола IP маршрутизатора \textbf{r1}.

\begin{Verbatim}
auto lo
iface lo inet loopback

auto eth0
iface eth0 inet static
	address 10.0.10.2
	netmask 255.255.255.0
	

auto eth1
iface eth1 inet static
	address 10.0.20.2
	netmask 255.255.255.0
	
	up ip r add 10.0.30.0/24 via 10.0.20.3 dev eth1
	down ip r del 10.0.30.0/24
	
	up ip r add 10.0.40.0/24 via 10.0.20.3 dev eth1
	down ip r del 10.0.40.0/24
\end{Verbatim}

Ниже приведён файл настройки протокола IP рабочей станции \textbf{r2}.

\begin{Verbatim}
auto lo
iface lo inet loopback

auto eth0
iface eth0 inet static
	address 10.0.20.3
	netmask 255.255.255.0
	
	up ip r add 10.0.10.0/24 via 10.0.20.2 dev eth0
	down ip r del 10.0.10.0/24

auto eth1
iface eth1 inet static
	address 10.0.30.1
	netmask 255.255.255.0
	
	up ip r add 10.0.40.0/24 via 10.0.30.2 dev eth1
	down ip r del 10.0.40.0/24
\end{Verbatim}

Ниже приведён файл настройки протокола IP рабочей станции \textbf{r3}.

\begin{Verbatim}
auto lo
iface lo inet loopback

auto eth0
iface eth0 inet static
	address 10.0.30.2
	netmask 255.255.255.0
	
	up ip r add 10.0.10.0/24 via 10.0.30.1 dev eth0
	down ip r del 10.0.10.0/24
	
	up ip r add 10.0.20.0/24 via 10.0.30.1 dev eth0
	down ip r del 10.0.20.0/24

auto eth1
iface eth1 inet static
	address 10.0.40.2
	netmask 255.255.255.0
\end{Verbatim}

Ниже приведён файл настройки протокола IP рабочей станции \textbf{ws1}.

\begin{Verbatim}
auto lo
iface lo inet loopback

auto eth0
iface eth0 inet static
	address 10.0.10.1
	netmask 255.255.255.0
	gateway 10.0.10.2
\end{Verbatim}


Ниже приведён файл настройки протокола IP рабочей станции \textbf{ws2}.

\begin{Verbatim}
auto lo
iface lo inet loopback

auto eth0
iface eth0 inet static
	address 10.0.20.1
	netmask 255.255.255.0
	gateway 10.0.20.3
\end{Verbatim}

Ниже приведён файл настройки протокола IP рабочей станции \textbf{ws3}.

\begin{Verbatim}
auto lo
iface lo inet loopback

auto eth0
iface eth0 inet static
	address 10.0.40.1
	netmask 255.255.255.0
	gateway 10.0.40.2
\end{Verbatim}

\section{Таблица маршрутизации}

Вывести (командой ip r) таблицу маршрутизации для \textbf{r1}.

\begin{Verbatim}
10.0.20.0/24 dev eth1  proto kernel  scope link  src 10.0.20.2 
10.0.30.0/24 via 10.0.20.3 dev eth1 
10.0.40.0/24 via 10.0.20.3 dev eth1 
10.0.10.0/24 dev eth0  proto kernel  scope link  src 10.0.10.2 
\end{Verbatim}

Вывести (командой ip r) таблицу маршрутизации для \textbf{r2}.

\begin{Verbatim}
10.0.20.0/24 dev eth0  proto kernel  scope link  src 10.0.20.3 
10.0.30.0/24 dev eth1  proto kernel  scope link  src 10.0.30.1 
10.0.40.0/24 via 10.0.30.2 dev eth1 
10.0.10.0/24 via 10.0.20.2 dev eth0 
\end{Verbatim}

Вывести (командой ip r) таблицу маршрутизации для \textbf{r3}.

\begin{Verbatim}
10.0.20.0/24 via 10.0.30.1 dev eth0 
10.0.30.0/24 dev eth0  proto kernel  scope link  src 10.0.30.2 
10.0.40.0/24 dev eth1  proto kernel  scope link  src 10.0.40.2 
10.0.10.0/24 via 10.0.30.1 dev eth0 
\end{Verbatim}

Вывести (командой ip r) таблицу маршрутизации для \textbf{ws1}.

\begin{Verbatim}
10.0.10.0/24 dev eth0  proto kernel  scope link  src 10.0.10.1 
default via 10.0.10.2 dev eth0 
\end{Verbatim}

Вывести (командой ip r) таблицу маршрутизации для \textbf{ws2}.

\begin{Verbatim}
10.0.20.0/24 dev eth0  proto kernel  scope link  src 10.0.20.1 
default via 10.0.20.3 dev eth0 
\end{Verbatim}

Вывести (командой ip r) таблицу маршрутизации для \textbf{ws3}.

\begin{Verbatim}
10.0.40.0/24 dev eth0  proto kernel  scope link  src 10.0.40.1 
default via 10.0.40.2 dev eth0
\end{Verbatim}

% ... Повторять для всех маршрутизаторов и рабочих станций, где есть что-то кроме gateway.

\section{Проверка настройки сети}

Вывод \textbf{traceroute} от ws3 до ws2 при нормальной работе сети.

\begin{Verbatim}
traceroute to 10.0.20.1 (10.0.20.1), 64 hops max, 40 byte packets
 1  10.0.40.2 (10.0.40.2)  5 ms  0 ms  0 ms
 2  10.0.30.1 (10.0.30.1)  13 ms  0 ms  1 ms
 3  10.0.20.1 (10.0.20.1)  15 ms  2 ms  1 ms
\end{Verbatim}

Вывод \textbf{traceroute} от ws2 до r1(eth0) при нормальной работе сети.

\begin{Verbatim}
traceroute to 10.0.10.2 (10.0.10.2), 64 hops max, 40 byte packets
 1  10.0.20.3 (10.0.20.3)  0 ms  0 ms  0 ms
 2  10.0.10.2 (10.0.10.2)  23 ms  1 ms  1 ms
\end{Verbatim}

Вывод \textbf{traceroute} от ws1 до r2(eth1) при нормальной работе сети.

\begin{Verbatim}
traceroute to 10.0.30.1 (10.0.30.1), 64 hops max, 40 byte packets
 1  10.0.10.2 (10.0.10.2)  4 ms  1 ms  0 ms
 2  10.0.30.1 (10.0.30.1)  12 ms  1 ms  1 ms
\end{Verbatim}


\section{Маршрутизация}

% На пути здесь достаточно быть одному аршрутизатору!

Вначале стоит написать, какие MAC-адреса интерфейсов в опыте были у каких машин.
Затем вывести маршрутную таблицу маршрутизатора (вывод команды ip r!)

от ws1(eth0 - 10.0.10.1/24)
до r2(eth0 - 10.0.30.1/24)
через r1(eth0 - 10.0.10.2/24  и eth1 - 10.0.20.2/24)

маршрутная таблица маршрутизатора r1
\begin{Verbatim}
10.0.20.0/24 dev eth1  proto kernel  scope link  src 10.0.20.2 
10.0.30.0/24 via 10.0.20.3 dev eth1 
10.0.40.0/24 via 10.0.20.3 dev eth1 
10.0.10.0/24 dev eth0  proto kernel  scope link  src 10.0.10.2 
\end{Verbatim}

Показаны опыты после стирания кеша ARP.
% Не забудьте это сделать!
На ws1 будет вызвана команда:
\begin{Verbatim}
ping 10.0.30.1
\end{Verbatim}

Далее показана отправка пакета на маршрутизатор r1(косвенная маршрутизация). 

\begin{Verbatim}
tcpdump -tne -i eth0

a6:f9:52:b6:1e:69 > ff:ff:ff:ff:ff:ff, ethertype ARP (0x0806), length 42: arp who-has 10.0.10.2 tell 10.0.10.1
0e:ab:f8:0c:10:4b > a6:f9:52:b6:1e:69, ethertype ARP (0x0806), length 42: arp reply 10.0.10.2 is-at 0e:ab:f8:0c:10:4b
a6:f9:52:b6:1e:69 > 0e:ab:f8:0c:10:4b, ethertype IPv4 (0x0800), length 98: 10.0.10.1 > 10.0.30.1: ICMP echo request, id 18690, seq 1, length 64
0e:ab:f8:0c:10:4b > a6:f9:52:b6:1e:69, ethertype IPv4 (0x0800), length 98: 10.0.30.1 > 10.0.10.1: ICMP echo reply, id 18690, seq 1, length 64
0e:ab:f8:0c:10:4b > a6:f9:52:b6:1e:69, ethertype ARP (0x0806), length 42: arp who-has 10.0.10.1 tell 10.0.10.2
a6:f9:52:b6:1e:69 > 0e:ab:f8:0c:10:4b, ethertype ARP (0x0806), length 42: arp reply 10.0.10.1 is-at a6:f9:52:b6:1e:69
\end{Verbatim}

Затем маршрутизатор отправил его далее на маршрутизатор r2.

\begin{Verbatim}
tcpdump -tne -i eth0

fa:de:dc:30:96:57 > ff:ff:ff:ff:ff:ff, ethertype ARP (0x0806), length 42: arp who-has 10.0.20.3 tell 10.0.20.2
3a:40:ee:31:9e:cd > fa:de:dc:30:96:57, ethertype ARP (0x0806), length 42: arp reply 10.0.20.3 is-at 3a:40:ee:31:9e:cd
fa:de:dc:30:96:57 > 3a:40:ee:31:9e:cd, ethertype IPv4 (0x0800), length 98: 10.0.10.1 > 10.0.30.1: ICMP echo request, id 18690, seq 1, length 64
3a:40:ee:31:9e:cd > fa:de:dc:30:96:57, ethertype IPv4 (0x0800), length 98: 10.0.30.1 > 10.0.10.1: ICMP echo reply, id 18690, seq 1, length 64
3a:40:ee:31:9e:cd > fa:de:dc:30:96:57, ethertype ARP (0x0806), length 42: arp who-has 10.0.20.2 tell 10.0.20.3
fa:de:dc:30:96:57 > 3a:40:ee:31:9e:cd, ethertype ARP (0x0806), length 42: arp reply 10.0.20.2 is-at fa:de:dc:30:96:57
\end{Verbatim}

\section{Продолжительность жизни пакета}

Для создания маршрутной петли (сеть 10.0.30.0/24 будет завёрнута между r2 и r1) на маршрутизаторе r2 были запущены следующие команды:

\begin{Verbatim}
ip l set eth1 down
ip r add 10.0.30.0/24 via 10.0.20.2 dev eth0
\end{Verbatim}

Теперь на r2 таблица маршрутизации выглядит следующим образом:

\begin{Verbatim}
10.0.20.0/24 dev eth0  proto kernel  scope link  src 10.0.20.3 
10.0.30.0/24 via 10.0.20.2 dev eth0 
10.0.10.0/24 via 10.0.20.2 dev eth0 
\end{Verbatim}

С ws1 отправим ping в завёрнутую сеть:

\begin{Verbatim}
ping 10.0.30.2 -c 1
PING 10.0.30.2 (10.0.30.2) 56(84) bytes of data.
From 10.0.20.3 icmp_seq=1 Time to live exceeded
\end{Verbatim}

На r2 будем перехватывать трафик на интерфейсе, подключенном к завёрнутой сети:

\begin{Verbatim}
fa:de:dc:30:96:57 > 3a:40:ee:31:9e:cd, ethertype IPv4 (0x0800), length 98: (tos 0x0, ttl 63, id 0, offset 0, flags [DF], proto ICMP (1), length 84) 10.0.10.1 > 10.0.30.2: ICMP echo request, id 18178, seq 1, length 64
3a:40:ee:31:9e:cd > fa:de:dc:30:96:57, ethertype IPv4 (0x0800), length 98: (tos 0x0, ttl 62, id 0, offset 0, flags [DF], proto ICMP (1), length 84) 10.0.10.1 > 10.0.30.2: ICMP echo request, id 18178, seq 1, length 64
...
3a:40:ee:31:9e:cd > fa:de:dc:30:96:57, ethertype ARP (0x0806), length 42: arp who-has 10.0.20.2 tell 10.0.20.3
fa:de:dc:30:96:57 > 3a:40:ee:31:9e:cd, ethertype ARP (0x0806), length 42: arp reply 10.0.20.2 is-at fa:de:dc:30:96:57
\end{Verbatim}

В итоге, когда r1 должен был в очередной раз отправить пакет на r2,  TTL достигло значения 0, и  r1 отправил ICMP-сообщение с информацией о том, что время жизни пакета истекло.

\section{Изучение IP-фрагментации}

Уменьшим MTU сети 10.0.30.1/24. Для этого на r2 введём команду:

\begin{Verbatim}
ip l set dev eth1 mtu 576
Изменение MTU
\end{Verbatim}

А на r3 eth0:
\begin{Verbatim}
ip l set dev eth0 mtu 576
Изменение MTU
\end{Verbatim}

% Напоминаем, что PMTU следует отключить!

На ws2 отключим PMTU и запустим ping с размером пакета 1000 на ws3:

\begin{Verbatim}
echo 1 >/proc/sys/net/ipv4/ip_no_pmtu_disc
ping 10.0.40.1 -c 1 -s 1000
\end{Verbatim}

Вывод \textbf{tcpdump} на маршрутизаторе r2 перед сетью с уменьшенным MTU.

% Вывод в ширину можно и сократить, удалив несущественные моменты!

\begin{Verbatim}
IP (tos 0x0, ttl 64, id 57496, offset 0, flags [none], proto ICMP (1), length 1028) 10.0.20.1 > 10.0.40.1: ICMP echo request, id 18434, seq 1, length 1008
IP (tos 0x0, ttl 62, id 54250, offset 0, flags [none], proto ICMP (1), length 1028) 10.0.40.1 > 10.0.20.1: ICMP echo reply, id 18434, seq 1, length 1008
\end{Verbatim}

Вывод \textbf{tcpdump} на маршрутизаторе r3 после сети с уменьшенным MTU.

% Вывод в ширину можно и сократить, удалив несущественные моменты!

\begin{Verbatim}
IP (tos 0x0, ttl 62, id 57496, offset 0, flags [none], proto ICMP (1), length 1028) 10.0.20.1 > 10.0.40.1: ICMP echo request, id 18434, seq 1, length 1008
IP (tos 0x0, ttl 64, id 54250, offset 0, flags [none], proto ICMP (1), length 1028) 10.0.40.1 > 10.0.20.1: ICMP echo reply, id 18434, seq 1, length 1008
\end{Verbatim}


Вывод \textbf{tcpdump} на узле получателя ws3.

\begin{Verbatim}
IP (tos 0x0, ttl 62, id 57496, offset 0, flags [none], proto ICMP (1), length 1028) 10.0.20.1 > 10.0.40.1: ICMP echo request, id 18434, seq 1, length 1008
IP (tos 0x0, ttl 64, id 54250, offset 0, flags [none], proto ICMP (1), length 1028) 10.0.40.1 > 10.0.20.1: ICMP echo reply, id 18434, seq 1, length 1008
\end{Verbatim}


\section{Отсутствие сети}

C ws1 была запущена команда:

\begin{Verbatim}
ping 10.0.100.1 -c 1
PING 10.0.100.1 (10.0.100.1) 56(84) bytes of data.
From 10.0.10.2 icmp_seq=1 Destination Net Unreachable
\end{Verbatim}

 На маршрутизаторе r1 запущен перехват трафика:
 
\begin{Verbatim}
tcpdump -n -i eth0

14:07:58.214405 IP 10.0.10.1 > 10.0.100.1: ICMP echo request, id 19202, seq 1, length 64
14:07:58.214423 IP 10.0.10.2 > 10.0.10.1: ICMP net 10.0.100.1 unreachable, length 92
\end{Verbatim}

Как видно, производится только один ARP-запрос, на который тут же генерируется ICMP-сообщение с информацией о том, что искомая сеть не найдена.

\section{Отсутствие IP-адреса в сети}

C ws1 была запущена команда:

\begin{Verbatim}
ping 10.0.30.11 -c 1
PING 10.0.30.11 (10.0.30.11) 56(84) bytes of data.
From 10.0.20.3 icmp_seq=1 Destination Host Unreachable
\end{Verbatim}

На маршрутизаторе r2 запущен перехват трафика на интерфейсе, подключённом к той же сети, что и ws1:

\begin{Verbatim}
tcpdump -n -i eth0

09:20:03.950262 IP 10.0.10.1 > 10.0.30.11: ICMP echo request, id 17922, seq 1, length 64
09:20:06.954618 IP 10.0.20.3 > 10.0.10.1: ICMP host 10.0.30.11 unreachable, length 92
\end{Verbatim}

На маршрутизаторе r2 запущен перехват трафика на интерфейсе, подключённом к сети, в котором должен был находится целевой ip-адрес:

\begin{Verbatim}
tcpdump -n -i eth1

09:22:52.750690 arp who-has 10.0.30.11 tell 10.0.30.1
09:22:53.744466 arp who-has 10.0.30.11 tell 10.0.30.1
09:22:54.744442 arp who-has 10.0.30.11 tell 10.0.30.1

\end{Verbatim}

Как видно, производится 3 ARP-запроса с таймаутом в 1 секунду. После того, как на последний запрос в течении секунды не пришёл ответ, 
генерируется ICMP-сообщение с информацией о том, что целевой IP-адрес не найден.

\end{document}
